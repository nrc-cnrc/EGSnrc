
%%%%%%%%%%%%%%%%%%%%%%%%%%%%%%%%%%%%%%%%%%%%%%%%%%%%%%%%%%%%%%%%%%%%%%%%%%%%%%%
%
%  EGSnrc manual: radiation transport
%  Copyright (C) 2015 National Research Council Canada
%
%  This file is part of EGSnrc.
%
%  EGSnrc is free software: you can redistribute it and/or modify it under
%  the terms of the GNU Affero General Public License as published by the
%  Free Software Foundation, either version 3 of the License, or (at your
%  option) any later version.
%
%  EGSnrc is distributed in the hope that it will be useful, but WITHOUT ANY
%  WARRANTY; without even the implied warranty of MERCHANTABILITY or FITNESS
%  FOR A PARTICULAR PURPOSE.  See the GNU Affero General Public License for
%  more details.
%
%  You should have received a copy of the GNU Affero General Public License
%  along with EGSnrc. If not, see <http://www.gnu.org/licenses/>.
%
%%%%%%%%%%%%%%%%%%%%%%%%%%%%%%%%%%%%%%%%%%%%%%%%%%%%%%%%%%%%%%%%%%%%%%%%%%%%%%%
%
%  Author:          Iwan Kawrakow, 2003
%
%  Contributors:    Blake Walters
%                   Ernesto Mainegra-Hing
%
%%%%%%%%%%%%%%%%%%%%%%%%%%%%%%%%%%%%%%%%%%%%%%%%%%%%%%%%%%%%%%%%%%%%%%%%%%%%%%%


\section{Radiation transport in EGSnrc}
\label{section_2}

%\subsection{General discussion}
\subsection{Introduction}
% Replace commented line for the one with fixed date when commiting
% Beware: Using the macro below conflicts between CVS and latex!!!
% \lfoot[{\sffamily {\leftmark}}]{{\small Last edited $Date: 2011/05/02 18:40:33 $
\lfoot[{\sffamily {\leftmark}}]{{\small Last edited 2011/05/02 18:32:38
}}


Photons interact with surrounding matter via four
basic processes: materialization into an electron/positron pair
in the electromagnetic field of the nuclei and surrounding
atomic electrons, incoherent (Compton) scattering with
atomic electrons,
photo-electric absorption and
coherent (Rayleigh) scattering with
the molecules (or atoms) of the medium. The first three
collision types transfer energy from the photon radiation
field to electrons\footnote{In this report, we often refer to
both positrons and electrons as simply {\it electrons}.
Distinguishing features will be brought out in the context.},
one of them dominate depending on energy and the medium
in which the transport takes place. The pair production
process\footnote{Occasionally the materialization into an $e^+e^-$ pair
takes place with the participation of an atomic electron which, after
receiving sufficient energy, is set free. Such processes are known
as triplet production.} dominates at high energies.
At some intermediate energies incoherent scattering is the most
important process, at low energies the photo-electric process
dominates.

Electrons, as they traverse matter, lose energy by two basic processes:
inelastic collisions with atomic electrons and radiation.
Radiative energy loss, which occurs in form of bremsstrahlung and
positron annihilation, transfers energy back to photons
and leads to the coupling of the electron and photon
radiation fields. The bremsstrahlung process is the dominant
mechanism of electron energy loss at high energies, inelastic collisions
are more important at low energies.
In addition, electrons participate in elastic collisions with atomic
nuclei which occur at a high rate and lead to frequent changes
in the electron direction.

Inelastic electron collisions and photon interactions with atomic electrons
lead to excitations and ionizations of the atoms along the paths of
the particles. Highly excited atoms, with vacancies in inner shells,
relax via the emission of photons and electrons with
characteristic energies.

The coupled integro-differential equations
that describe the electromagnetic shower development are
prohibitively complicated to allow for an analytical treatment
except under severe approximations. The Monte Carlo (MC) technique
is the only known solution method that can be applied for
any energy range of interest.

Monte Carlo
simulations of particle transport processes are a faithful simulation
of physical reality: particles are ``born" according to distributions
describing the source, they travel certain distances, determined by a
probability distribution depending on the total interaction
cross section, to the site of a collision and scatter into
another energy and/or direction according to the corresponding
differential cross section, possibly producing new
particles that have to be transported as well.
This procedure is continued until all particles are
absorbed or leave the geometry under consideration.
Quantities of interest can be calculated by averaging over a given set of
MC particle ``histories" (also refereed to as ``showers'' or
``cases''). From mathematical points of view each particle
``history'' is one point in a $d$-dimensional space (the dimensionality
depends on the number of interactions) and the averaging procedure
corresponds to a $d$-dimensional Monte Carlo integration.
As such, the Monte Carlo estimate of quantities of interest
is subject to a
statistical uncertainty which depends on
$N$, the number of particle histories simulated,
and usually decreases as $N^{-1/2}$. Depending on the
problem under investigation and the desired statistical
accuracy, very long calculation times may be necessary.

An additional difficulty occurs in the case of the Monte Carlo
simulation of electron transport. In the process of slowing down,
a typical fast electron
and the secondary particles it creates
undergo hundreds of thousands of interactions with
surrounding matter.  Because of this large number of collisions,
an event-by-event simulation of electron transport is often
not possible due to limitations in computing power.  To
circumvent this difficulty, Berger \cite{Be63} developed the
``condensed history" (CH) technique for the simulation
of charged particle transport. In this method, large numbers of
subsequent transport and collision
processes are ``condensed" to a single ``step''
The cumulative effect of the individual interactions is taken
into account by sampling the change of the particle's energy,
direction of motion, and position, at the end of the step from appropriate
multiple scattering distributions.
The CH technique, motivated
by the fact that single collisions with the atoms cause, in most cases,
only minor changes in the particle's energy and direction of flight,
made the MC simulation of charged particle transport possible but introduced
an artificial parameter, the step-length. The dependence of the calculated
result on the step-length has become known as a step-size artifact
\cite{BR89}.

EGSnrc is a general purpose package for the Monte Carlo
simulation of coupled electron and photon transport that
employs the CH technique.
It is based on the popular EGS4 system \cite{Ne85}
but includes a variety of enhancements in the CH implementation
and in some of the underlying cross sections. We recognize that many
of the modifications that we have made to the original
EGS4 implementation are not important for
high energy applications, initially EGS4' primary target.
On the other side, the energy range of application of the
EGS4 system has shifted over the years to lower and lower
energies. To facilitate this transition many enhancements
to the original EGS4 implementation has been developed,
{\em e.g.} the PRESTA algorithm \cite{BR87}, the
inclusion of angular distribution of bremsstrahlung
photons \cite{Bi89}, the low energy photon cross section
enhancements by the group at KEK/Japan \cite{Na98}, to
mention only some of them. The availability of these
improvements, recent advances in the theoretical understanding
of the condensed history technique \cite{KB97a,Ka99a} and
multiple elastic scattering \cite{KB97}, as well as
unpublished results of our recent research have
motivated us to undertake a major re-work of the EGS4
system the result of which is EGSnrc.

It is the purpose of this report to summarize the current stage of the
EGSnrc system. We have attempted a self-consistent presentation
and so, some of the material contained in this report is
not new. In particular, various parts come from the EGS4 manual, SLAC-265
by Nelson et al\cite{Ne85}.
\index{SLAC-265}
\index{Nelson, Ralph}

This report does not attempt to provide a complete treatment of Monte
Carlo methods or probability and sampling theory.
Readers not familiar with the Monte Carlo technique are encouraged to
read one of the many excellent reviews available.
